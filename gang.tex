%%% Local Variables:
%%% mode: latex
%%% TeX-master: t
%%% End:

\documentclass{article}
\usepackage{amsmath,amsthm,hyperref}
\pagestyle{empty}

\setlength{\parindent}{0pt}
\setlength{\parskip}{\baselineskip}

\newtheorem*{question}{Question}

\begin{document}

A {\em job\/} is a list of $m$ versions of uniformly-sized rectangular
printed pieces, represented as an $m$-tuple of positive integers:
\[
  V = (v_1, \ldots , v_m)
\]

An $n$-up {\em form\/} is an arrangement of $n$ uniformly-sized
rectangular images on a single printed press sheet that can be cut
into individual pieces after printing. A form is represented as an
$m$-tuple of non-negative integers: $ (f_1, \ldots, f_m) $, where each
$f_i$ is the number of copies of version $v_i$ on the press sheet and
$\sum{f_i} = n$. Then the set of all possible $n$-up forms for $m$
versions is given by
\[
  F_{m,n} = \{ (f_1,\ldots,f_m) \mid \sum_{i=1}^{m} f_i = n \}
\]
Note that the number of such forms is the number of weak
compositions\footnote{see \url{https://en.wikipedia.org/wiki/Composition_(combinatorics)}.}
of $n$ in $m$ terms, giving
\[
  |F_{m,n}| = \binom{m+n-1}{n-1}
\]

An $N$-tuple $(x_1,\ldots,x_N)$, where $N = |F_{m,n}|$, is a {\em solution\/} for job $V$ if
\[
  \sum_{i=1}^{N} x_i f_j \geq v_j \text{, for all }\, 1 \leq j \leq m
\]
Conceptually, $x_i$ is the quantity of the press run of the $i$th form.

Let the cost of a press run of quantity $x > 0$ be defined as $A x +
B$, where $A$ is the cost per sheet and $B$ is the fixed set up cost
of a press run. Then the cost function of a solution is
\[
  c(x_1,\ldots,x_N) =
  \begin{cases}
    Ax_i + B & x > 0 \\
    0        & x = 0
  \end{cases}
\]

\begin{question}
Given a job $V$ run $n$-up with cost per sheet $A$  and set up cost
$B$ per press run, what solution $(x_1,\ldots,x_N)$ has the minimum
cost?
\end{question}


\end{document}
